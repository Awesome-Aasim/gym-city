\documentclass{article}
\usepackage[dvipsnames]{xcolor}
\usepackage{pgfplots}
\usepackage{adjustbox}
\usepackage{appendix}
\pgfplotsset{compat=1.10}
\newcount\recurdepth

\newcommand\storelabel[2]{\expandafter\xdef\csname label#1\endcsname{#2}}
\newcommand\getlabel[1]{\csname label#1\endcsname}
\newcommand{\absval}[1]{\ifnum#1<0 -\fi#1}

% if you need to pass options to natbib, use, e.g.:
%     \PassOptionsToPackage{numbers, compress}{natbib}
% before loading neurips_2019

% ready for submission
% \usepackage{neurips_2019}

% to compile a preprint version, e.g., for submission to arXiv, add add the
% [preprint] option:
%     \usepackage[preprint]{neurips_2019}

% to compile a camera-ready version, add the [final] option, e.g.:
%    \usepackage[final]{neurips_2019}

% to avoid loading the natbib package, add option nonatbib:
%     \usepackage[nonatbib]{neurips_2019}

\usepackage[utf8]{inputenc} % allow utf-8 input
\usepackage[T1]{fontenc}    % use 8-bit T1 fonts
\usepackage{hyperref}       % hyperlinks
\usepackage{url}            % simple URL typesetting
\usepackage{booktabs}       % professional-quality tables
\usepackage{amsfonts}       % blackboard math symbols
\usepackage{amsmath}
\usepackage{nicefrac}       % compact symbols for 1/2, etc.
\usepackage{microtype}      % microtypography
\usepackage{tikz}
\usetikzlibrary{arrows}
\usetikzlibrary{matrix}
\usetikzlibrary{positioning}
\usepackage{bnumexpr}
\usepackage{forloop}
\usepackage{xcolor}
\usepackage{xfp}

\title{Fractal Neural Networks as Ensembles of Continuous-Valued Cellular Automata: Pathfinding and Related Problems of Variable Scale}

\begin{document}
\pgfplotsset{every axis plot/.append style={each nth point=1}}
\pgfplotsset{possible/.style={double, ultra thick, samples=100, color =PineGreen}}
\pgfplotsset{width=13cm, height=4cm}


\section{Game of Life Unit Activation Function}

\begin{align*}
	\boldsymbol{\mathrm{w}} = \begin{bmatrix}
		1& 1& 1\\
		1& 9& 1\\
		1& 1& 1\\
	\end{bmatrix}
\end{align*}
% GoLU
\begin{tikzpicture}[every mark/.append style={mark size=1.5pt}]
    \pgfmathsetlengthmacro\MajorTickLength{\pgfkeysvalueof{/pgfplots/major tick length} * 2
		}
\begin{axis}[domain  = 0:18,
         axis x line = center,
         axis y line = center,
         samples = 100,
         xmin    = 0,
         xmax    = 18,
         ymin    = -0,
         ymax    = 1,
         xtick={},
         ytick={1},
         xtick   = {},
         xlabel  = {$\boldsymbol{\mathrm{w}}\cdot\boldsymbol{\mathrm{s_{t-1}}}$},
         ylabel  = {$s_{0}$},
         set layers,
         every axis plot/.append style={blue},
         x label style={
             anchor=west,},
         y label style={
             anchor=south,
         },
         major tick length = \MajorTickLength,
         every tick/.style={
         black,
     semithick,},
         minor tick num = 0,
         legend cell align=left,
         legend style={fill=white, 
             fill opacity=0.6, text opacity=1, draw=none, font=\footnotesize,
             at={(0.2,1,0)}, anchor=north west},
         grid = both,
               ]
							\addplot[domain= -23:0, possible] {0};
							\addplot[domain= 0:2, possible, forget plot] {0};
				\addplot[possible, domain= 1:5, forget plot] coordinates{
			(1, 0)
			(2, 0)
			(3, 1)
			(4, 0)
			(5, 0)
			(6, 0) 
            (7, 0) 
            (8, 0)
            (9, 0)
            (10, 0)
            (11, 1)
            (12, 1)
            (13, 0)
            (17.5, 0)
	};

 	\addplot[only marks, color=Mahogany, opacity=1, mark options={scale=1.4}, samples=100, domain= 1:5, mark=*] coordinates{
        (0,0)
        (1,0)
        (2,0)
        (3,1)
        (4,0)
        (5,0)
        (6,0)
        (7,0)
        (8,0)
        (9,0)
        (10,0)
        (11,1)
        (12,1)
        (13,0)
        (14,0)
        (15,0)
        (16,0)
        (17,0)
 	};
%				\addplot[only marks, samples=100, domain= 0:1, color=green!50!gray, opacity=0.5] coordinates{
%					(1, 1)
%					(1/2, 1/2)
%					(1/4, 1/4)
%					(1/8, 1/8)
%					(1/16, 1/16)
%					(1/32, 1/32)
%					(1/64, 1/64)
%					(1/128, 1/128)
%					(1/256, 1/256)
%					(1/512, 1/512)
%
%	};
\addlegendentry{possible nonlinearity}
\addlegendentry{discrete GoL activations}
\addlegendentry{'' in channel $U$}
\addlegendentry{'' in $W$}
\end{axis}
\end{tikzpicture}

%
%
%\begin{tikzpicture}[every mark/.append style={mark size=1.5pt}]
%    \pgfmathsetlengthmacro\MajorTickLength{\pgfkeysvalueof{/pgfplots/major tick length} * 2
%		}
%  \begin{axis}[domain  = -21:35,
%         axis x line = center,
%         axis y line = center,
%         samples = 100,
%         xmin    = -20,
%         xmax    = 36,
%         ymin    = -.5,
%         ymax    = 1.5,
%         xtick={},
%         ytick={1},
%         xtick   = {-20, -16, -12, -8, -4, 0, 1, 6, 12, 18, 23, 26, 29, 32},
%         xlabel  = {$a$},
%         ylabel  = {$g(a)$},
%         set layers,
%         every axis plot/.append style={blue},
%         x label style={
%             anchor=west,},
%         y label style={
%             anchor=south,
%         },
%         major tick length = \MajorTickLength,
%         every tick/.style={
%         black,
%     semithick,},
%         minor tick num = 4,
%         legend cell align=left,
%         legend style={fill=white, 
%             fill opacity=0.6, text opacity=1, draw=none, font=\footnotesize,
%             at={(-.05,1.3,0)}, anchor=north west},
%         grid = both,
%               ]
%							\addplot[domain= -23:0, possible] {0};
%							\addplot[domain= 0:1, possible, forget plot] {x};
%				\addplot[possible, domain= 1:5, forget plot] coordinates{
%			(1, 1)
%			(2, 1)
%			(3, 1)
%			(4, 1)
%			(5, 1)
%			(6, 0) (7, 1) (11, 1) 
%			(12, 0) (13, 1) (17, 1) 
%			(18, 0) (19, 1)
%			(20, 1)
%			(35, 1)
%	};
%							 \addplot[samples=100, only marks, domain=-20:0, mark=*, opacity=0.5] coordinates{
%							 (-20, 0) 
%						   (-16, 0) 
%							 (-15, 0) 
%						   (-12, 0) (-11, 0) 
%						(-10, 0) 
%					(-8, 0) (-7, 0) (-6, 0) (-5, 0) 
%				(-4, 0) (-3, 0) (-2, 0) (-1, 0) (0, 0)
%			(20,1)
%			(23, 1)
%			(24, 1)
%			(26, 1)
%			(27, 1) (28, 1)
%			(29, 1)
%			(30, 1) (31, 1) 
%			(32, 1)
%			(33, 1) (34, 1) (35, 1) (36, 1)
%
%			(1, 1)
%			(2, 1)
%			(3, 1)
%			(4, 1)
%			};
%
%	\addplot[only marks, color=red, opacity=0.5, samples=100, domain= 1:5, mark=*] coordinates{
%		(-6,0) (-5, 0) (-4, 0) (-3,-0) (-2, 0) (-1,0)
%		(0,0) 
%			(1, 1)
%			(2, 1)
%			(3, 1)
%			(4, 1)
%			(6, 0) (7, 1) (8, 1) (9, 1) 
%			(12, 0) (13, 1) (14, 1)
%			(18, 0) (19, 1)
%			(1/2, 1/2)
%					(3/4, 3/4)
%					(5/8, 5/8)
%					(11/16, 11/16)
%					(21/32, 21/32)
%					(43/64, 43/63)
%	};
%				\addplot[only marks, samples=100, domain= 0:1, color=green!50!gray, opacity=0.5] coordinates{
%					(1, 1)
%					(1/2, 1/2)
%					(1/4, 1/4)
%					(1/8, 1/8)
%					(1/16, 1/16)
%					(1/32, 1/32)
%					(1/64, 1/64)
%					(1/128, 1/128)
%					(1/256, 1/256)
%					(1/512, 1/512)
%
%	};
%\addlegendentry{possible nonlinearity $g(a)$}
%\addlegendentry{activations in $R$ and $S$}
%\addlegendentry{'' in channel $U$}
%\addlegendentry{'' in $W$}
%\end{axis}
%\end{tikzpicture}
%
 \end{document}

